\documentclass[conference]{IEEEtran}
% \IEEEoverridecommandlockouts
% The preceding line is only needed to identify funding in the first footnote. If that is unneeded, please comment it out.
\usepackage{cite}
\usepackage{amsmath,amssymb,amsfonts}
\usepackage{algorithmic}
\usepackage{graphicx}
\graphicspath{{images/}}
\usepackage{textcomp}
\usepackage{xcolor}
\usepackage[raggedrightboxes]{ragged2e}
\def\BibTeX{{\rm B\kern-.05em{\sc i\kern-.025em b}\kern-.08em
    T\kern-.1667em\lower.7ex\hbox{E}\kern-.125emX}}
\begin{document}

\title{Pathfinding Robot}

\author{
	\IEEEauthorblockN{James Bao}
	\IEEEauthorblockA{\textit{Dept. of Electrical, Computer,}\\ \textit{and Software Engineering} \\
		\textit{The University of Auckland}\\
		Auckland, New Zealand \\
		jbao577@aucklanduni.ac.nz}
	\and
	\IEEEauthorblockN{Benson Cho}
	\IEEEauthorblockA{\textit{Dept. of Electrical, Computer,}\\ \textit{and Software Engineering} \\
		\textit{The University of Auckland}\\
		Auckland, New Zealand \\
		bcho892@aucklanduni.ac.nz}
	\and
	\IEEEauthorblockN{Nicholas Wolf}
	\IEEEauthorblockA{\textit{Dept. of Electrical, Computer,}\\ \textit{and Software Engineering} \\
		\textit{The University of Auckland}\\
		Auckland, New Zealand \\
		nwol626@aucklanduni.ac.nz}
	\and
	\IEEEauthorblockN{Viktor Neshikj}
	\IEEEauthorblockA{\textit{Dept. of Electrical, Computer,}\\ \textit{and Software Engineering} \\
		\textit{The University of Auckland}\\
		Auckland, New Zealand \\
		vnes637@aucklanduni.ac.nz}
}

\maketitle

\begin{abstract}
	Our Pathfinding Robot is an intelligent, two-wheeled robot capable of navigating the shortest-path between points-of-interest within a maze, projected as a line from a ceiling mounted projector.
	This robot has been developed in COMPSYS 301 during Semester Two of 2023 by Team 1, and has involved the exploration of both hardware and firmware/software aspects.
	This development has included time- and frequency-domain analysis of phototransistor and photodiode sensors, analogue circuit design, sensor constellation \& printed circuit board design, pathfinding algorithm implementation, and firmware development on a PSoC 5LP microcontroller.
	This engineering effort has produced a high-quality, robust robot that comfortably exceeds the goals that we set out to achieve.
	The engineering design considerations and decisions made towards this outcome are outlined in this following report.
\end{abstract}

\begin{IEEEkeywords}
	psoc 5, pathfinding, photodiode, firmware, pcb
\end{IEEEkeywords}



\section{Hardware Design}

\subsection{Frequency Domain Analysis}

\subsection{Analogue Circuitry Design}

\subsection{Printed Circuit Board Design}



\section{Pathfinding Algorithms}



\section{Firmware Implementation}



\section{Integration}



\section*{Acknowledgment}

A big thank you to our lecturers Dr. Morteza Biglari-Abhari and Dr. Maryam Hemmati; Teaching Assistants Ross Porter, Asher Butler, James Park, and Callum Iddon.

\section*{References}

Coursebook information from COMPSYS 305 directed by Dr. Morteza Biglari-Abhari of the Department of Electrical, Computer, and Software Engineering at The University of Auckland.

\nocite{*}
\bibliographystyle{../IEEEtran}
\bibliography{annot}

\end{document}
